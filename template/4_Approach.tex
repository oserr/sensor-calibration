\section{Approach}
\label{sec:approach}
From a high level perspective, our approach to explore the nuances of enabling
interrupts on the PowerDue is very simple, because it consists of three distinct
aspects:

\begin{enumerate}
\item Enabling the PowerDue to be interrupted by creating and registering an
      interrupt handler with the PowerDue.
\item Calibrating the readings from the sensor to obtain threshold values that
      could be applied to the interrupts, to prevent being interrupted by the
      continuously changing magnetic field in any given environment.
\item Configuring the FXOS8700CQ to interrupt PowerDue whenever it detects
      magnetic events above the threshold values.
\end{enumerate}

The three axes threshold values were computed by

\begin{align}
    x_{th} &= \lvert \overline{x} \rvert + \sigma_{x} * m \\
    y_{th} &= \lvert \overline{y} \rvert + \sigma_{y} * m \\
    z_{th} &= \lvert \overline{z} \rvert + \sigma_{z} * m
\end{align}

where $x_{th}$, $y_{th}$, and $z_{th}$ are the threshold values, $\overline{x}$,
$\overline{y}$, and $\overline{z}$ are the averages of the readings,
$\sigma_{x}$, $\sigma_{y}$, and $\sigma_{z}$ are the standard deviations of the
readings, and $m$ represents a constant multiplier to prevent the sensor from
interrupting on changes that can be expected from the continuous variation in
the magnetic field.
