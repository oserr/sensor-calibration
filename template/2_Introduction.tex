\section{Introduction}
\label{sec:intro}
Power efficiency is crucial for wireless embedded systems, because energy is a
limiting resource and it can be expensive or impractical to replace batteries in
wireless devices. Interrupts are a form of hardware-software mechanism that can
enable an event-driven architecture, making a system not only more power
efficient, but even more accurate than a polling architecture, because a polling
architecture will be oblivious to events that occur whenever there is no poll.

This work explores the nuances of enabling interrupts on the PowerDue, by
setting up the FXOS8700CQ sensor to interrupt the PowerDue when it detects a
magnetometer event. Ultimately, an interrupt-driven application  can result in a
more complex architecture, because there is very tight coupling between the
hardware and software; however, such an architecture will improve the power
efficiency of the system because it allows the idle task to run for longer
periods of time, or even to throttle down the clock rate, consuming less CPU
cycles, and it allows the application to yield more accurate readings, because
it does not miss events that would otherwise be missed in a polling architecture
whenever such a system is in between polls.
