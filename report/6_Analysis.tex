\section{Analysis}
\label{sec:analysis}

\emph{In this section, step back and look at the data you've collected and the trendlines you've identified.  What does it mean?  Interpret the data.  Does the trend match your prediction?  If not, why not?  Look at the measurements and compare them to what you predicted in the Approach section, point-by-point.  Can you really draw the conclusions in light of the measurement error?  How could you have done better?}

It is suspicious that we recorded zero flight time for zero payload weight.  The birds all flew a finite distance.  This would imply speeds in excess of the speed of light.  I think the subjects to whom I delegated this measurement had too much to drink that afternoon, and I'll have to go back and engage in some kingly attitude adjustment.

Imagining that I had plotted air-speed vs. weight (I got away with this because I am the king--I don't recommend that you try it), it would be clearer that there is an inverse square-law trend, supporting our hypothesis. Of interest is that the absolute magnitude of the error in the measurements increases as the value itself increases, suggesting that the bulk of the error is not from constant sources, such as human reaction time, but from variables that would be affected by the duration of the flight. We hypothesize it could be related to the number of times the bird has to stop during flight; a heavier weight results in more rest stops but not necessarily the same number of stops per run.

We can thus confidently say that the air-speed velocity of both African and European swallows vary as the inverse-square of the load.  I, Arthur, King of the Britons declare this model to be valid forevermore and give it herewith the name GRAIL: Generic Radio-Avian Intercommunication Latency.

Also, the \emph{"African or European?"} question has been put to rest once and for all.  Myth busted.
