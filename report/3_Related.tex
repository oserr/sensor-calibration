\section{Related Work}
\label{sec:related}
The iOS energy efficiency guide \cite{iOSEnergy} provides a high level view of
how event-driven architectures can improve the energy efficiency of the iOS
phone. Despite the fact that the material is geared toward iOS app development,
the material is relevant becuase it applies generally to wireless systems, and
thus inspires our motivation for implemening an event-driven architecture. More
specifically, \cite{CargoNet} shows how an interrupt-driven architectures, a
specific realization of an event-driven architecture, can lead to more power
efficient and accurate WSNs, providing motivation for the interrupt-driven
approach of our application.

To allow the PowerDue to be interrupted by the sensor, the Arduino reference for
\textit{attachInterrupt} \cite{InterruptRef} was relevant, because it shows the
API of the function, and provides an example of how an interrupt handler can be
registered with the PowerDue to handle interrupts on a given input PIN. Simply
attaching an interrupt handler to the PowerDue is not enough to configure the
system, because FXOS8700CQ also has to be configured to send an interrupt signal
to the PowerDue, and \cite{DatasheetSPI} specifies how magnetometer registers
have to be configured to enable FXOS8700CQ to send interrupt signals.
