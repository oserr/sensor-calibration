%\documentclass[onecolumn]{IEEEtranTIE}
\documentclass[journal]{IEEEtranTIE}
\usepackage{graphicx}
\usepackage{cite}
\usepackage{picinpar}
\usepackage{amsmath}
\usepackage{url}
\usepackage{flushend}
\usepackage[latin1]{inputenc}
\usepackage{colortbl}
\usepackage{soul}
\usepackage{multirow}
\usepackage{pifont}
\usepackage{color}
\usepackage{alltt}
\usepackage[hidelinks]{hyperref}
\usepackage{enumerate}
\usepackage{siunitx}
\usepackage{breakurl}
\usepackage{epstopdf}
\usepackage{pbox}

\begin{document}
\title{	ConAuth - Context for Authentication \\ (Nov. 2017)}

\author{
	\vskip 1em
	{
	Saurabh Sharma, \emph{saurabh.sharma@sv.cmu.edu}\\
	Omar Serrano, \emph{omar.serrano@sv.cmu.edu}
	}
}

\maketitle

\begin{abstract}
With the growing number of wireless devices, we need efficient mechanisms to let
the wireless devices communicate securely. The wireless devices sometimes share
common sensors that can be leveraged to perform additional authentication
procedures on a set of localized wireless devices. The problem which prevents
such a judicious use of sensors is the orientation of wireless devices. Sensors
such as gyroscope and accelerometer are commonly found in wireless devices, but
their readings make no sense until their orientations are the same. We plan to
conduct controlled experiments to investigate how different environmental factors
impact the accelerometer performance and how the best accuracy can be achieved
in an appropriate condition range. We also characterize the nature of an
accelerometer to understand its performance in different conditions. Based on
such comprehensive understanding, we propose to estimate the phone attitude  and
provide  for opportunistic calibration of the accelerometer
\end{abstract}

\begin{IEEEkeywords}
Contextual security, sensor funsion, Madgwick.
\end{IEEEkeywords}

\definecolor{limegreen}{rgb}{0.2, 0.8, 0.2}
\definecolor{forestgreen}{rgb}{0.13, 0.55, 0.13}
\definecolor{greenhtml}{rgb}{0.0, 0.5, 0.0}

\section{Introduction}

\IEEEPARstart{W}{ith}
growing number of IoT devices, securely pairing a new device into an existing
set of devices is an extremely important yet burdensome task. Traditionally,
these devices are paired manually, where an operator sets up an authentication
with the existing network of devices. Specifically, we address the problem of a
platoon ghost attack wherein an attacker device spoofs presence within a platoon
to gain admission and subsequently execute malicious attacks \cite{Han}. To
address such concerns, we present ConAuth, a novel autonomous admission scheme
which bindsu the devices to their physical context (i.e., locality). ConAuth
exploits the findings that devices in a local setting experience similar context
to prove to each other over time that they are co-present \cite{Han}.
Specifically, they experience similar events(e.g., people coming inside the
room, knocking on the door). Our approach is based on the ability of the devices
to capture this context, using sensors that both the devices share. We design
and implement the ConAuth protocol and evaluate a proof-of-concept
implementation using a set of experiments. Our implementation will demonstrate
that devices in the same room can be sufficiently distinguished by their context
and this can be utilized to thwart platoon ghost attacks and similar misbehavior.

\section{Problem Statement}

With near-ubiquitous availability of wireless devices equipped with a wide
variety of sensors, research in building context aware services has been
growing. However, despite a large number of services proposed and developed as
research prototypes, the number of truly context-aware applications available in
wireless devices is quite limited. A major barrier for the large-scale
proliferation of context aware applications is poor accuracy \cite{Alanezi}. We
address one of the key reasons for this poor accuracy, which is the impact of
sensor orientation. Devices have their sensors oriented in different positions.
We first show that smartphone positions significantly affect the values of the
sensor data being collected by a context aware application, and this in turn
has a significant impact on the accuracy of the application \cite{Alanezi}.
Next, it describes the design and prototype development of a orientation
discovery service that accurately detects a sensor orientation. This service is
based on the sensor data collected from carefully chosen sensors. Finally, the
paper demonstrates that the accuracy of an existing context aware service or
application is significantly enhanced when run in conjunction with the proposed
orienttion discovery service.


\begin{figure}[!t]\centering
	\includegraphics[width=8.5cm]{phoneOrientation}
	\caption{Orientation for a mobile phone}\label{fig1}
\end{figure}


\section{Technical Approach}

Find the geoframe orientation of the PowerDue and mobile phone

\subsection{Compute attitude of device}

The phone attitude is obtained by doing continuos integration on the angular
velocity, and by taking the difference between the geo-framea, the earth
coordinate sytem, and the body-frame, the coordinate system of the device's
body. Per \cite{PhoneAttitude}, the best approach for calculating the difference
between both coordinate systems is the Euler Axis/Angle method, which solves the
problem from the geo-frames perspective. To compute the integration, the total
device motion is split into multiple time windows, and the rotation of the
device is the accumulated rotation of all the time windows.

\begin{figure}[!t]\centering
	\includegraphics[width=5.5cm]{acceleration}
	\caption{User acceleration}\label{fig2}
\end{figure}


\subsection{Madgwick}

Historically, the Kalman filter, and other techniques, including fuzzy
processing and frequency domain filters, have been used as the basis for
orientation algorithms; however, these techniques have several disadvantages
\cite{Madgwick}. For example, the Kalma filter requires is computationally
expensive, and some of the other techniques are only effective under limited
conditions \cite{Madgwick}. Therefore, we plan to use Madgwick, an algorithm
that employs a quaternion representation of orientation, which has had positive
results despite the fact that it does not require the heavy computational load,
or high sampling frequency, of a Kalman-filter based algorithm \cite{Madgwick}.

\section{Plan of Attack}

\begin{enumerate}
\item Obtain a sensor with a gyroscope and accelerometer.
\item Write the code for a driver, if necessary.
\item Compute the attitude of the PowerDue.
\item Compute the attidue of a mobile phone.
\item Correlate the readings from the PowerDue and mobile phone.
\end{enumerate}

\begin{figure}[!t]\centering
	\includegraphics[width=5.5cm]{magnetic_field}
	\caption{User acceleration}\label{fig3}
\end{figure}

\section{Conclusion}

Conclusion.

% References

\bibliographystyle{Bibliography/IEEEtranTIE}
\bibliography{Bibliography/IEEEabrv,Bibliography/BIB_1x-TIE-2xxx}\ %IEEEabrv instead of IEEEfull

\end{document}
